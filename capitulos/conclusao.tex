% ----------------------------------------------------------
% Conclusão
% ----------------------------------------------------------

\chapter{Conclusão}

% O desenvolvimento do presente trabalho foi de grande importância para aumentar os conhecimentos do tema proposto. Analisando aspectos relacionados as redes sociais e dispositivos móveis, usando especialmente essas abordagens no contexto do veganismo e vegetarianismo. Desse modo, entender a relação dessas variáveis foi muito útil para a compreensão e desenvolvimento do aplicativo no decorrer do trabalho.

% O trabalho teve como objetivo geral o desenvolvimento de um aplicativo de rede social, com o proposito de compartilhar conhecimentos e informações de receitas culinárias, principalmente pelo publico vegetariano e vegano. 


O trabalho teve como objetivo geral o desenvolvimento de um aplicativo de rede social para compartilhamento de conhecimentos e informações de receitas culinárias, principalmente pelo publico vegetariano e vegano, que demonstrou ser de grande importância para aumentar os conhecimentos acerca do tema proposto. Analisando aspectos relacionados as redes sociais e dispositivos móveis, usando especialmente essas abordagens no contexto do veganismo e vegetarianismo. Desse modo, entender a relação dessas variáveis foi muito útil para a compreensão e desenvolvimento do aplicativo no decorrer do trabalho.

Foi feita uma imersão no contexto abordado, possibilitando o desenvolvimento do aplicativo mobile, onde foi posto em balança se o mesmo seria desenvolvido de forma nativa ou hibrida usando a tecnologia do framework Ionic. Pensando de uma forma ágil para o que estava sendo proposto, conclui-se que era mais viável, inclusive pelo custo e tempo de desenvolvimento, que a utilização da tecnologia hibrida seria a solução mais adequada, onde supriria todas as necessidades durante o desenvolvimento e abrangeria as plataformas mobiles mais utilizadas.

Ao final do desenvolvimento, aplicativo passou por testes e atendeu bem as expectativas, gerando resultados satisfatórios, concluindo que o problema relatado sobre as dificuldade de encontrar variedades de receitas em um só local foi solucionada com a criação do aplicativo, permitindo assim, que os objetivos propostos fossem realmente alcançados.

Para finalizar, a partir do que foi desenvolvido para este trabalho,  é possível realizar pesquisas mais elaboradas para ter uma visão mais detalhada, criando estrategias para crescimento do aplicativo. 

% Como o aplicativo desenvolvido trata-se de um MVP (\textit{Minimum Viable Product}), dentre futuros aprimoramentos, os usuários poderão tirar duvidas por mensagens com nutricionistas e educadores físicos que possam vir a ser parceiros do aplicativo, inserção de vídeos e slides de fotos ao adicionar uma receita, compartilhar as receitas em outras redes sociais, notificações push.
