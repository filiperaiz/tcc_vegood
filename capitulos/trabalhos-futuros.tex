% ----------------------------------------------------------
% Trabalhos Futuros
% ----------------------------------------------------------

%% Para tornar os trabalhos futuros um capítulo a parte, use \chapter{Trabalhos Futuros}
\chapter{Trabalhos Futuros}

% Para trabalho futuro, uma proposta seria o desenvolvimento de aplicativos nativos para dispositivos móveis.

A aplicação desenvolvida neste trabalho trata-se de um MVP (\textit{Minimum Viable Product}). Dentre outros futuros aprimoramentos, lista-se : 

\begin{lista}
   
     \item \textbf{Estudo de mercado}:  para ter uma visão mais detalhada do nicho desejado e criar estratégias de ação para o produto;
      \item \textbf{Versões nativas}: desenvolvimento de versões nativas mais aprimoradas do aplicativo;
% 	\item \textbf{Publicação do Aplicativo}: registro junto às maiores lojas de aplicativos mobiles, Google Play e Apple Store, para que a aplicação possa ser publicada;
	\item \textbf{Envio de notificações Push}: funcionalidade necessária para que a interação entre a aplicação e o usuário possa crescer;
	\item \textbf{Vídeos e slide de fotos}: possibilitar inserir vídeos ou mais de uma foto na criação de uma receita;
	\item \textbf{Compartilhamento das receitas}: funcionalidade que irá permitir que os usuários compartilhem suas receitas publicadas no aplicativo na sua página do Facebook, aumentando a visibilidade da aplicação para os amigos que ainda não possuem.;
	\item \textbf{Acompanhamento nutricional}: Os usuários
	poderão tirar duvidas por mensagens com nutricionistas e educadores físicos que possam vir a ser parceiros do aplicativo.
\end{lista}
