%% Resumo
\begin{resumo}
As redes sociais tem como propósito o compartilhamento de conhecimentos e informações. Com a grande popularização da internet as redes sociais se virtualizaram, atendendo as necessidades de comunicação e os relacionamentos da vida real usando o ambiente virtual. Em paralelo a evolução da internet e das redes sociais, outra area que obteve grande crescimento foi o vegetarianismo e veganismo. Grande parte dos adeptos deste regime alimentar iniciam essa prática devido as questões relacionadas a saúde, economia, ambiente, ética e religião. Em busca de uma alimentação adequada, muitos seguidores optam em começar a fazer suas próprias refeições ao invés buscarem em restaurantes. O presente trabalho apresenta o desenvolvimento de um aplicativo mobile implementado na plataforma híbrida Ionic, que produz código executável para os ambientes operacionais IOS 7+ e Android 4.2+, além de integrada ao Facebook. O aplicativo irá funcionar como uma rede social, onde os usuários poderão publicar receitas, disponibilizá-las para todos que estiverem usando o aplicativo. Poderão curtir, favoritar e comentar na receita publicada como também seguir quem publicou a receita, criando assim, um vínculo social dentro do aplicativo.

\vspace{\onelineskip}
\noindent\\
\textbf{Palavras-chaves}: Redes sociais, vegetarianismo, veganismo, aplicativo mobile, Ionic, IOS, Android, Facebook.
\end{resumo}